%% Преамбула TeX-файла

% По госту положен 14-ый шрифт, однако мы возьмём 12-ый.

% 1. Стиль и язык
%\documentclass [12pt,oneside]{rusthesis} % Стиль (по умолчанию 14pt)
\documentclass[pdftex,12pt,a4paper]{report}

\usepackage[T2A]{fontenc}
\usepackage[utf8]{inputenc}
\usepackage[english, russian]{babel}

% 2. Обязательные вещи
\renewcommand{\labelitemi}{\normalfont\bfseries{--}} % список без номеров
\newcommand{\HRule}{\rule{\linewidth}{0.5mm}}
\newcommand{\code}[1]{\texttt{#1}}

\sloppy

% 3. Добавляем гипертекстовое оглавление в PDF
\usepackage[
bookmarks=true, colorlinks=true, unicode=true,
urlcolor=black,linkcolor=black, anchorcolor=black, c
itecolor=black, menucolor=black, filecolor=bla,
]{hyperref}


% Прочее, дополнять по вкусу
\usepackage{graphicx}	% Пакет для включения рисунков

%\usepackage{makecell}
%\usepackage{multirow}
%\usepackage{ulem}
%\usepackage{indentfirst}

%\usepackage[pdftex]{graphicx}
%\graphicspath{{pic/}}
%\usepackage{tikz}

% Увы, поля придётся уменьшить из-за листингов.
\topmargin -1cm
\oddsidemargin -0.5cm
\evensidemargin -0.5cm
\textwidth 17cm
\textheight 24cm

% Картнки и tikz
%\usetikzlibrary{snakes,arrows,shapes}



\Russian 


\begin{document}
\begin{titlepage}
 
\begin{center}
 
 
% Upper part of the page

 
\textsc{\LARGE МГТУ им. Н.Э.Баумана}\\[1.5cm]
 
\textsc{\Large Домашнее задание  по ПВС}\\[0.5cm]
 
 

 
% Author and supervisor
\begin{minipage}{0.4\textwidth}
\begin{flushleft} \large
\emph{Студенты:}\\
Варламов Е.Д. \\
Кашин И.С.
\end{flushleft}
\end{minipage}
\begin{minipage}{0.4\textwidth}
\begin{flushright} \large
\emph{Преподователь:} \\
Крищенко В. А.
\end{flushright}
\end{minipage}
 
\vfill
 
% Bottom of the page
{\large \today}
 
\end{center}
 
\end{titlepage}

\tableofcontents

\chapter{Введение}
Цель работы.\\ Создание POP3 сервера , поддерживающего команды \code{USER, STAT, LIST, RETR, DELE, RSET, QUIT}\\
Вариант 3.\\ Для работы используется один поток и pselect, для логирования -- второй поток, связь по PosixMQ. 

\chapter{Проектирование реализации протокола}
\section{Автомат}

\begin{figure}
\center{\includegraphics[width=1\linewidth]{./automata.png}}
\caption{Автомат разбора}
\label{automata:image}
\end{figure}

\section{Взаимодействие подсистем}
\begin{figure}
\center{\includegraphics{./ssystems.png}}
\caption{Граф взаимодействия подсистем}
\label{ssystems:image}
\end{figure}

\chapter{Опиcание программной реализации}
\section{Графы вызова функций}
\subsection{Полный граф вызовов}
\begin{figure}
\center{\includegraphics[width=1\linewidth]{./everything_scf.png}}
\caption{Полный граф вызовов}
\label{everything:image}
\end{figure}

\chapter{Описание процесса сборки}
В данной главе описан процесс сборки проекта. Используется система сборки \code{Make}. В корневом каталоге проекта находится \code{Makefile}, содержащий цели для управления компиляцией, тестированием и генерацией отчёта. Этот \code{Makefile} использует несколько дополнительных скриптов и довольно большую логику объвязки. Основные цели:

\begin{enumerate}
\item \code{clean} --- очистка дирректорий \code{build} и \code{bin}.
\item \code{tests} --- прогонка системы тестирования и генерация куска отчёта, этому посвящённая.
\item \code{report} --- сборка отчёта. 
\end{enumerate}

Таким образом, для сборки всего проекта с нуля необходимо в корневом каталоге (\code{trunk}) выполнить сначала \code{make}, а потом \code{make tests \&\& make report}. Первый скомпилит исполняемый файл, вторая пара соберёт отчёт. В дальнейшем можно использовать только вторую пару комманд, чтобы не пересобирать всё с нуля, или \code{make clean \&\& make \&\& make tests \&\& make report} для пересборки и тестирования в случае изменений в коде.

\begin{figure}
\center{\includegraphics[width=1\linewidth]{./make_main.png}}
\caption{Граф по основному Make-файлу}
\label{make_main:image}
\end{figure}

\begin{figure}
\center{\includegraphics[width=1\linewidth]{./make_report.png}}
\caption{Граф по Make-файлу для генерации отчёта}
\label{make_report:image}
\end{figure}

\chapter{Результаты системного тестирования}
Логи тестов:
\input{scenarios.tex}

\end{document}

